\documentclass[%
	11pt,
	a4paper,
	utf8,
	%twocolumn
		]{article}	

\usepackage{style_packages/podvoyskiy_article_extended}


\begin{document}
\title{Теория алгоритмов, структуры данных и сопряженные вопросы}

\author{}

\date{}
\maketitle

\thispagestyle{fancy}

\tableofcontents

\section{Задачи}

\subsection{Объединение последовательностей целых чисел в дипазоны}

Пусть есть массив неотрицатетльных целых чисел \verb|[3, 0, 1, 2, 5, 8, 9, 15]|. Требуется построить диапазоны вида \verb|"0-3,5,8-9,15|.

\begin{lstlisting}[
style = ironpython,
numbers = none
]
import typing as t

def seq_to_ranges(seq: t.Sequence[int]) -> str:
    """Объединяет элементы последовательности в диапазоны"""
    
    # принцип раннего отказа
    # если последовательность пустая, возбуждаем исключение
    assert seq, "Error! Empty sequence ..."
    
    if len(seq) == 1:
        return str(seq[0])
    
    seq: t.List[int] = sorted(set(seq))
    
    groups: t.List[t.List[int]] = []
    group: t.List[int] = [seq[0]]
    
    for value in seq:
        if value == group[-1]:
            continue
            
        if (value - group[-1]) == 1:
            group.append(value)
        # если разность != 1, значит группа закончилась
        else:
            groups.append(group)
            group = [value]
   groups.append(group)  # NB! 
            
   _result: t.List[str] = []
   for group in groups:
       if len(group) > 1:
           _result.append(f"{group[0]}-{group[-1]}")
       else:
           _result.append(str(group[0]))
   
   return ",".join(_result)

# Использование
seq = (7,)
seq_to_ranges(seq)  # '7'

seq = (1, 2, 3)
seq_to_ranges(seq)  # '1-3'

seq = (1, 5, 2, 8, 15, 16, 20, 17)
seq_to_ranges(seq)  # '1-2,5,8,15-17,20'

seq = [1, 2, 3, 5, 6, 7, 10, 11, 12, 15, 20]
seq_to_ranges(seq)  # '1-3,5-7,10-12,15,20'
\end{lstlisting}

\subsection{Удаление элемента из массива за линейное время}

Дан массив целых чисел. Требуется удалить заданный элемент \url{https://programforyou.ru/poleznoe/how-to-remove-values-from-array-effectively}

Решение на Python
\begin{lstlisting}[
style = ironpython,
numbers = none
]
import typing as t

def remove_elem(seq: t.Sequence[int], remove_value: int) -> t.List[int]:
    """Удаляет элемент по значению"""
    # в Python параметры в функцию можно передать только по соиспользованию;
    # на практике это означает, что формальные параметры функции всегда получают
    # копии ссылок на фактические аргументы, поэтому если объект изменяемый, то
    # его можно изменить из-под функции;
    # чтобы не изменять список в глобальной области видимости,
    # приходится создавать копию списка с помощью list()
    seq: t.List[int] = list(seq)
    
    if (value not in seq):
        raise ValueError(
            f"Ошибка! Указанное значение ({value}) "
            "не встречается среди элементов списка"
        )
        
    j = 0
    for value in seq:
        # если текущий элемент не совпадает с удаляемым, то элемент 
        # помещается на свое же место и индекс инкрементируется;
        # в противном случае индекс элемента не инкрементируется и потому
        # следующий элемент встает на место удаленного
        if (value != remove_value):
            seq[j] = value
            j += 1
    
    # итоговый список будет очевидно на один элемент короче,
    # поэтому нужно забрать все элементы кроме последнего
    rerturn seq[:-1]
        
# Использование
seq = [10, 8, -5, 6, 0, 3]
remove_elem(seq, remove_value=6)  # [10, 8, -5, 0, 3]

seq = (5, 3, 0, 6)
remove_elem(seq, remove_value=3)  # [5, 0, 6]
\end{lstlisting}




% Источники в "Газовой промышленности" нумеруются по мере упоминания 
\begin{thebibliography}{99}\addcontentsline{toc}{section}{Список литературы}
	\bibitem{koltzov-c-lang:2019}{ \emph{Кольцов Д.М.} Си на примерах. Практика, практика и только практика. -- СПб.: Наука и Техника, 2019. -- 288 с.}
\end{thebibliography}

%\listoffigures\addcontentsline{toc}{section}{Список иллюстраций}

\lstlistoflistings\addcontentsline{toc}{section}{Список листингов}

\end{document}
