\documentclass[%
	11pt,
	a4paper,
	utf8,
	%twocolumn
		]{article}	

\usepackage{style_packages/podvoyskiy_article_extended}


\begin{document}
\title{Теория алгоритмов и сопряженные вопросы}

\author{}

\date{}
\maketitle

\thispagestyle{fancy}

\tableofcontents

\section{Задачи}

\subsection{Удаление элемента из массива за линейное время}

Дан массив целых чисел. Требуется удалить заданный элемент \url{https://programforyou.ru/poleznoe/how-to-remove-values-from-array-effectively}

Решение на Python
\begin{lstlisting}[
style = ironpython,
numbers = none
]
import typing as t

def remove_elem(array: t.List[int], value: int) -> t.List[int]:
    """Удаляет элемент по значению"""
    if (value not in set(array)):
        raise ValueError(
            f"Ошибка! Указанное значение ({value}) не встречается среди элементов списка"
        )
        
    j = 0
    # в Python параметры в функцию можно передать только по соиспользованию;
    # на практике это означает, что формальные параметры функции всегда получают
    # копии ссылок на фактические аргументы, поэтому если объект изменяемый, то
    # его можно изменить из-под функции;
    # чтобы не изменять список в глобальной области видимости, приходится создавать копию списка
    array = array[:]  
    for elem in enumerate(array):
        if (elem != value):
            array[j] = elem
            j += 1
        
        rerturn array[:idx]
        
def main():
    array = [10, 8, -5, 6, 0, 3]
    remove_elem(array, value=6)  # [10, 8, -5, 0, 3]
    
if __name__ == "__main__":
    main()
\end{lstlisting}




% Источники в "Газовой промышленности" нумеруются по мере упоминания 
\begin{thebibliography}{99}\addcontentsline{toc}{section}{Список литературы}
	\bibitem{koltzov-c-lang:2019}{ \emph{Кольцов Д.М.} Си на примерах. Практика, практика и только практика. -- СПб.: Наука и Техника, 2019. -- 288 с.}
\end{thebibliography}

%\listoffigures\addcontentsline{toc}{section}{Список иллюстраций}

\lstlistoflistings\addcontentsline{toc}{section}{Список листингов}

\end{document}
